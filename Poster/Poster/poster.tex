\documentclass[b1paper,portrait]{template/baposter}

\usepackage[font=small,labelfont=bf]{caption} % Required for specifying captions to tables and figures
\usepackage{booktabs} % Horizontal rules in tables
\usepackage{relsize} % Used for making text smaller in some places
\usepackage{amsmath,amsfonts,amssymb,amsthm} % Math packages
\usepackage{eqparbox}
\usepackage{multicol}
\usepackage{textcomp}
\usepackage{graphicx,wrapfig,lipsum}
\usepackage{wrapfig}
\graphicspath{{graphics/}} % Directory in which figures are stored

\newtheorem{theorem}{Teorema}
\newtheorem{lemma}{Lema}
\newtheorem*{definition}{Definicija}

 \definecolor{bordercol}{RGB}{20,112,1} % Border color of content boxes
 \definecolor{headercol1}{RGB}{20,112,1} % Background color for the header in the content boxes (left side)
 \definecolor{headercol2}{RGB}{20,112,1} % Background color for the header in the content boxes (right side)
 \definecolor{headerfontcol}{RGB}{255,255,255} % Text color for the header text in the content boxes
 \definecolor{boxcolor}{RGB}{240 240 255} % Background color for the content in the content boxes

\topmargin -3cm
\renewcommand{\baselinestretch}{0.9}

\begin{document}

\background
{
	\begin{tikzpicture}[remember picture,overlay]
		\draw (current page.north west)+(-2em,2em) node[anchor=north west]
		{\includegraphics{Suma}};
	\end{tikzpicture}
}

\begin{poster}
{
	grid=false,
	borderColor=bordercol, % Border color of content boxes
	headerColorOne=headercol1, % Background color for the header in the content boxes (left side)
	headerColorTwo=headercol2, % Background color for the header in the content boxes (right side)
	headerFontColor=headerfontcol, % Text color for the header text in the content boxes
	boxColorOne=boxcolor, % Background color for the content in the content boxes
	headershape=roundedright, % Specify the rounded corner in the content box headers
	headerfont=\Large\sf\bf, % Font modifiers for the text in the content box headers
	textborder=rectangle,
	background=user,
	headerborder=open, % Change to closed for a line under the content box headers
	boxshade=plain,
	columns=6
} {\includegraphics[scale=0.17]{PetLogo}}
%
%
%!!!!!!!!!!!!!!!!!!!!!!!!!!!!!!!!!!!!!!!!!!!!!!!!!!!!!!!!!!
%
%
{\small \color{white} 
	\textbf{XVI konferencija polaznika Istra\v ziva\v cke stanice Petnica "Korak u nauku" \\
	\bf  \huge Problem izlaska iz \v sume}
}
{\color{white} 
	\textbf{\smaller $^1$Katarina Krivoku\' ca, $^2$Dimitrije Gluk\v cevic\\
		   \footnotesize $^1$katarinakrivokuca@outlook.com,$^2$dimchee90@gmail.com \\
		  \footnotesize \textit{$^1$Matemati\v cka gimnazija, Beograd, $^2$ Gimnazija ,,Svetozar Markovic", Ni\v s}\\
		   \smaller mentor: Bojan Ba\v si\' c, PMF Novi Sad}
}
{\includegraphics[scale=0.29]{MatLogo}}


%!!!!!!!!!!!!!!!!!!!!!!!!!!!!!!!!!!!!!!!!!!!!!!!!!!!!!!!!!!


\headerbox{Problem}{name=uvod,column=0,row=0, span=3}
{ 
	\textbf{Definicija:}
 	Izgubili ste se sa prijateljima u n-dimenzionoj \v sumi.
 	Sve \v sto znate su oblik i dimenzije \v sume.
 	Kako najbr\v ze neko od vas mo\v ze da iza\dj e po pomo\' c?
 	\\
 	\\
 	\textbf{Koje \v sume mi posmatramo?}
 	\begin{itemize}
 		\item traka, (slika zalgallera...)
 		\item k-traka
 	\end{itemize}
   	\vspace{0.16cm}
}


\headerbox{Rezultati - trake u $\mathbb{R}^n$}{name=Rez1, column=0, row=0, span=3, below=uvod}
{
	\begin{itemize}
 		\item Ho\' cemo da na\dj emo neki, \v sto "manji" objekat koji ne mo\v zemo da smestimo u traku u $\mathbb{R}^n$
 		\item Teorema za veli\v cinu simpleksa, mo\v zda slika tetraedra izme\dj u ravni
 		\item Ho\' cemo da na\dj emo neku \v sto kra\v cu \v setnju po ovom objektu
 		\item Gornje ograni\v cenje za 1 \v coveka, slika tog puta na tetraedru
 		\item za n+1 ljudi slika tetraedar
 	\end{itemize}
 	%\begin{wrapfigure}{r}{3.5cm}\vspace*{-0.5cm}
	%\includegraphics[scale=0.1]{slika}
	%\end{wrapfigure}
	 \vspace{0.15cm}
}


\headerbox{Za\v sto je bolje i\' ci sa prijateljima?}{name=Rez2, column=3, row=0, span=3}
{
	\begin{itemize}
		\item Ovde ka\v zemo da je vi\v se ljudi na\v se uop\v stenje
		\item Def puta za k ljudi
		\item Sli\v cica Zalgallera i Y da ka\v zemo da je na\v se bolje
	\end{itemize}
	\vspace{0.15cm}
}


\headerbox{Rezultati - k-trake}{name=Rez3, column=3, row=0, span=3, below=Rez2}
{
	\begin{itemize}
		\item Ho\' cemo druga\v cije da uop\v stimo trake u vi\v sim dimenzijama
		\item Def k-trake, slika valjka u 3d
		\item Teorema za k-trake
	\end{itemize}
	\vspace{0.15cm}
}


\headerbox{Rezultati - donje ograni\v cenje}{name=Rez4, column=3, row=0, span=3, below=Rez3}
{
	\begin{itemize}
		\item Donje ograni\v cenje u $\mathbb{R}^n$
		\item Hipoteza da je i njihovo i na\v se donje ograni\v cenje slabo
	\end{itemize}
	\vspace{0.15cm}
}

\headerbox{Dalji rad}{name=Dalje, column=3, row=0, span=3, below=Rez4}
{
	\begin{itemize}
		\item Na\v se putanje za jednog \v coveka mogu da se pobolj\v saju
		\item Hipoteza $$\lim_{n \to \infty}S_n = ...$$
	\end{itemize}
	\vspace{0.15cm}
}

\end{poster}

\end{document}