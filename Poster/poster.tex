\documentclass[b1paper,portrait]{template/baposter}

\usepackage[font=small,labelfont=bf]{caption} % Required for specifying captions to tables and figures
\usepackage{booktabs} % Horizontal rules in tables
\usepackage{relsize} % Used for making text smaller in some places
\usepackage{amsmath,amsfonts,amssymb,amsthm} % Math packages
\usepackage{eqparbox}
\usepackage{multicol}
\usepackage{textcomp}
\usepackage{graphicx,wrapfig,lipsum}
\usepackage{wrapfig}
\graphicspath{{graphics/}} % Directory in which figures are stored

\newtheorem{theorem}{Teorema}
\newtheorem{lemma}{Lema}
\newtheorem*{definition}{Definicija}
\newtheorem*{hipoteza}{Hipoteza}

 \definecolor{bordercol}{RGB}{40,40,40} % Border color of content boxes
 \definecolor{headercol1}{RGB}{20,112,1} % Background color for the header in the content boxes (left side)
 \definecolor{headercol2}{RGB}{20,112,1} % Background color for the header in the content boxes (right side)
 \definecolor{headerfontcol}{RGB}{255,255,255} % Text color for the header text in the content boxes
 \definecolor{boxcolor}{RGB}{240 240 255} % Background color for the content in the content boxes

\topmargin -2.7cm
\renewcommand{\baselinestretch}{0.7}

\begin{document}

\background
{
	\begin{tikzpicture}[remember picture,overlay]
		\draw (current page.north west)+(-2em,2em) node[anchor=north west]
		{\includegraphics{Suma}};
	\end{tikzpicture}
}

\begin{poster}
{
	grid=false,
	borderColor=bordercol, % Border color of content boxes
	headerColorOne=headercol1, % Background color for the header in the content boxes (left side)
	headerColorTwo=headercol2, % Background color for the header in the content boxes (right side)
	headerFontColor=headerfontcol, % Text color for the header text in the content boxes
	boxColorOne=boxcolor, % Background color for the content in the content boxes
	headershape=roundedright, % Specify the rounded corner in the content box headers
	headerfont=\Large\sf\bf, % Font modifiers for the text in the content box headers
	textborder=rectangle,
	background=user,
	headerborder=open, % Change to closed for a line under the content box headers
	boxshade=plain,
	columns=6
} {\includegraphics[scale=0.15]{PetLogo}}
%
%
%!!!!!!!!!!!!!!!!!!!!!!!!!!!!!!!!!!!!!!!!!!!!!!!!!!!!!!!!!!
%
%
{\small \color{white} 
	\textbf{XIX konferencija polaznika Istra\v ziva\v cke stanice Petnica "Korak u nauku" \\
	\bf  \huge Problem izlaska iz \v sume}
}
{\color{white} 
	\textbf{
			\smaller Katarina Krivoku\' ca,  katarinakrivokuca@outlook.com, MATF Beograd\\
		   	Dimitrije Gluk\v cevi\' c, dimchee90@gmail.com, {Gimnazija "Svetozar Markovi\' c", Ni\v s}   \\
		   	mentor: Bojan Ba\v si\' c, PMF Novi Sad
		}
}
{\includegraphics[scale=0.25]{MatLogo}}


%!!!!!!!!!!!!!!!!!!!!!!!!!!!!!!!!!!!!!!!!!!!!!!!!!!!!!!!!!!


\headerbox{Problem}{name=uvod,column=0,row=0, span=3}
{ 
	\textbf{Postavka problema:}
 	Izgubili ste se sa prijateljima u $n$-dimenzionoj \v sumi.
 	Sve \v sto znate su oblik i dimenzije \v sume.
 	Kako najbr\v ze neko od vas mo\v ze da iza\dj e po pomo\' c?
 	\\
 	\\
 	\textbf{Koje \v sume mi posmatramo?}
 	\begin{itemize}
 		\item traka
 		\item $k$-traka
 		\item \v sume \v ciji je konveksni omota\v c trougao
 	\end{itemize}
 	{
 		%\begin{center}
		%{\includegraphics[scale=0.03]{Zalgaller}};
		%\end{center}
	}
   	\vspace{0.15cm}
}
\headerbox{Za\v sto je bolje i\' ci sa prijateljima?}{name=Rez2, column=0, row=0, span=3,below=uvod}
{
	Bavili smo se i na\v sim uop\v stenjem problema:
	\begin{itemize}
		\item Put za n ljudi je skup od $k$ puteva koji kre\' cu iz iste ta\v cke.
		\item Du\v zina puta za $n$ ljudi je suma du\v zina pojedina\v cnih puteva.
		\item Za put $P_k$ za $k$ ljudi ka\v zemo da je izlazan put iz \v sume $F$ ako za bilo koji po\v cetni polo\v zaj, barem jedan od n puteva iz skupa izlazi iz $F$.
	\end{itemize}
	%Sli\v cica Zalgallera i Y da ka\v zemo da je na\v se bolje
	\vspace{0.15cm}
}



\headerbox{Rezultati - trake u $\mathbb{R}^n$}{name=Rez1, column=3, row=0, span=3}
{
 	Ideja je bila  prona\' ci neki, \v sto ,,manji" objekat koji ne mo\v zemo da smestimo u traku u $\mathbb{R}^n$
 	\begin{theorem}
 		Izme\dj u dve hiperravni u $\mathbb{R}^n$ na udaljenosti 1 se ne mo\v ze postaviti pravilan $n$-simpleks stranice
 		$$\sqrt{\frac{2\left\lceil \frac{n}{2}\right\rceil\cdot \left( n-\left\lceil\frac{n}{2}\right\rceil+1\right)}{n+1}}$$
 		tako da ga ni jedna od ove dve hiperravni ne se\v ce. 
		%slika tetraedra izme\dj u ravni
 	\end{theorem}
 	Dalje smo tra\v zili \v sto kra\v cu \' setnju po ovom objektu

 	\begin{theorem} Put koji formiraju po trojkama nekoplanarne ivice $n$-simpleksa \v cije su ivice du\v zine 
 		$\sqrt{\frac{2\left\lceil \frac{n}{2}\right\rceil\cdot \left( n-\left\lceil\frac{n}{2}\right\rceil+1\right)}{n+1}}$
 		je izlazni put iz trake u n dimenzija i taj put ima du\v zinu:
		$$L_{n,1}=n\cdot \sqrt{\frac{2\left\lceil \frac{n}{2}\right\rceil\cdot \left( n-\left\lceil\frac{n}{2}\right\rceil+1\right)}{n+1}}.$$
		%slika tog puta na tetraedru
	\end{theorem}
	\begin{theorem}
 		Put koji formira $n+1$ du\v zi od centra do svih temena pravilnog $n$-simpleksa  \v cije su ivice du\v zine
 		$\sqrt{\frac{2\left\lceil \frac{n}{2}\right\rceil\cdot \left( n-\left\lceil\frac{n}{2}\right\rceil+1\right)}{n+1}}$
 		je izlazni put za $n+1$ ljudi iz trake u $n$ dimenzija i taj put ima du\v zinu:
		$$L_{n,n+1}=\sqrt{n\cdot\left\lceil\frac{n}{2}\right\rceil\cdot\left( n-\left\lceil\frac{n}{2}\right\rceil+1\right)}$$.
 		%slika tetraedar
 	\end{theorem}
 	%\begin{wrapfigure}{r}{3.5cm}\vspace*{-0.5cm}
	%\includegraphics[scale=0.1]{slika}
	%\end{wrapfigure}
	 \vspace{0.15cm}
}



\headerbox{Rezultati - k-trake}{name=Rez3, column=0, row=0, span=3, below=Rez2}
{
	\v Sta ako druga\v cije uop\v stimo trake u vi\v sim dimenzijama? \\ \\
	\begin{definition}
		Defini\v simo \textbf{k-traku u n dimenzija} za svako prirodno $n$ i za svako $1\leqslant k\leqslant n-1$  kao \v sumu odredjenu
		svim ta\v ckama iz $\mathbb{R}^n$ na udaljenosti manjoj ili jednakoj $\frac{1}{2}$ od nekog $k$-dimenzionog potprostora $\mathbb{R}^n$. 
	\end{definition}
	\begin{theorem}
		Najkra\' ca izlazna putanja iz $k$-trake u $\mathbb{R}^n$ je ista kao najkra\' ca izlazna putanja iz trake u $\mathbb{R}^{k+1}$.
	\end{theorem}
	 %slika valjka u 3d
	\vspace{0.15cm}
}


\headerbox{Rezultati - donje ograni\v cenje}{name=Rez4, column=0, row=0, span=3, below=Rez3}
{
	Gomi je pokazao gornje organi\v cenje za du\v zinu puta \v sirine 1.
	On nas je inspirisao da uradimo sli\v cno za na\v se uop\v stenje sa n ljudi.
	\begin{theorem}
		Neka je du\v zina najkra\' ceg izlaznog puta jedini\v cne \v sirine za $k\geqslant 2$ ljudi iz trake u $n\geqslant 2$ dimenzija $S_k$. Tada va\v zi:
		$$S_k\geqslant\frac{k}{2k-2}\cdot \sqrt{2.2782^2+9(n-2)}$$
	\end{theorem}
	\begin{hipoteza}
		Kako na\v se konstrukcijski dobijena re\v senja imaju ceo "red veli\v cine" ve\' cu du\v zinu nego 
		data donja ograni\v cenja pretpostavljamo da je mogu\' ce poo\v striti ove ocene. 
	\end{hipoteza}
	\vspace{0.15cm}
}

\headerbox{Dalji rad}{name=Dalje, column=3, row=0, span=3, below=Rez1}
{
	Posmatranjem objekata poput zarubljenih simpleksa, mogu\' ce je jo\v s pobolj\v sati predstavljene rezultate. Recimo za 3 dimenzije.\\ \\
	%slika zarubljenog tetraedra i puta po njemu
	Tako\dj e prirodno je postaviti i slede\' ce pitanje:
	"Grupa ljudi je zajedno izgubljena u \v sumi poznatih dimenzija. Koliko optimalno treba da ih bude,
	i koliki put moraju da pre\dj u, tako da znaju da \' ce iza\' ci iz \v sume?"
	\begin{hipoteza}
		Ako je za n-to dimenzionu \v sumu $F$ du\v zina najkra\' ceg izlaznog puta za $k$ ljudi $S_k(F)$, tada je:
		$$S(F)=\lim_{n \to \infty} S_n(F) = L_{n,n+1}$$
	\end{hipoteza}
	\vspace{0.15cm}
}

\end{poster}

\end{document}