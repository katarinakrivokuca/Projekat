\documentclass[11pt,letter]{article}
\usepackage{amsfonts,epsf,amsmath,amssymb,graphicx,tabularx,booktabs}
\usepackage{graphicx}
\usepackage{amsmath}
\usepackage{amsfonts}
\usepackage{amssymb}
\usepackage{booktabs}
\usepackage{tabularx}
\usepackage{epstopdf}
\usepackage[ruled,linesnumbered]{algorithm2e}
\usepackage{subfig}
\usepackage{multicol}
\usepackage{enumerate}
\usepackage{url}
\usepackage{float}
\usepackage{indentfirst}
\usepackage{babel}[serbian]
\usepackage{esvect}


\newtheorem{teo}{\bf Teorema}[section]
\newtheorem{posl}[teo]{\bf Posledica}
\newtheorem{lem}[teo]{\bf Lema}
\newtheorem{pred}[teo]{\bf Predlog}
\newtheorem{tvr}[teo]{\bf Tvr\dj enje}
\newtheorem{pro}[teo]{\bf Problem}
\newtheorem{df}[teo]{\bf Definicija}

\newcommand{\comment}[1]{\ $[\![${\normalsize #1}$]\!]$ \ }
\newcommand{\proof}{\noindent{\bf Proof.\ }}
\newcommand{\qed}{\hfill $\blacksquare$ \bigskip}
\def\cp{\,\square\,}



\renewcommand{\tablename}{Tabela}
\renewcommand{\figurename}{Slika}
\renewcommand{\refname}{Literatura}
\renewcommand{\abstractname}{Apstrakt}
\renewcommand{\proof}{\textbf{Dokaz.}}

% Centering the EPS graphics inside the FIGURE environment
\def\slika #1{\begin{center} \epsffile{#1} \end{center}}

\newenvironment{wlist}
{\vspace{-10pt}
\begin{list}{}
{\setlength{\labelwidth}{15mm} \setlength{\partopsep}{0pt} \setlength{\parskip}{0pt}
\setlength{\topsep}{0pt} \setlength{\itemsep}{0pt} \setlength{\parsep}{0pt}
\setlength{\labelsep}{10pt} \setlength{\leftmargin}{15mm}}
\item[]} {\end{list} \vspace{2pt}
\smallskip}

\textwidth 16cm \textheight 23.5cm \oddsidemargin 0.0cm \evensidemargin 0.0cm \voffset -2.2cm


\renewcommand{\thefootnote}{\fnsymbol{footnote}}
\begin{document}

\title{\bf{Projekat}}

\author{
Katarina Krivoku\'ca \\
Matemati\v cka gimnazija, Beograd, Srbija\\
e-mail: \tt{katarinakrivokuca@outlook.com} \\
\and
Dimitrije Gluk\v cevi\' c \\
Gimnazija ,,Svetozar Markovi\' c", Ni\v s, Srbija \\
e-mail: \tt{dimchee90@gmail.com} \\
}

\date{\today}
\maketitle %\vspace{-0.5cm}

\begin{abstract}
Belmanov problem tra\v zenja najkra\' ceg izlaznog puta iz \v sume poznatog oblika je poznato za mali broj \v suma, od kojih je jedna ograni\v cena sa dve paralelne prave na udaljenosti 1, tj. Zalgaller je prona\v sao najkra\' ci put \v sirine 1. Tako\dj e su istra\v zivani i najkra\' ci putevi \v sirine 1 u vi\v sim dimenzijama, Ghomi je pokazao gornje ograni\v cenje za du\v zinu takvog puta u $\mathbb{R}^3$ i donje ograni\v cenje u $\mathbb{R}^n$. U ovom radu \' cemo predstaviti gornje ogrni\v cenje za du\v zinu ovakvog puta u $\mathbb{R}^n$, kao i jednu modifikaciju Belmanovog problema u kojoj vi\v se ljudi kre\' ce iz iste ta\v cke te se posmatra skup puteva sa zajedni\v ckim po\v cetkom. Dalje je pokazano jedno gornje ograni\v cenje za du\v zinu najkra\' ceg puta \v sirine 1 u $\mathbb{R}^n$, kao i jedno gornje ograni\v cenje pri posmatranoj modifikaciji. Dosada\v snji rad i intuicija ukazuju na to da je pri posmatranoj modifikaciji u $\mathbb{R}^n$ put za $n+1$ ljudi pokazan u ovom radu najkra\' ci modifikovani put \v sirine 1.


\end{abstract}

\section[Uvod]{Uvod}
\bigskip
\subsection[Originalni problem]{Originalni problem}
\bigskip
Originalni problem "Izgubljeni u \v sumi" je postavio Belman 1956. i on je glasio:
\begin{center}
"Koji je najkra\' ci put kojim \v covek izgubljen u \v sumi poznatog oblika i dimenzija treba da ide tako da posle tog puta zna da \' ce sigurno iza\' ci iz nje?"
\end{center}
\indent Formalnije, problem se defini\v se kao "Koji je najkra\' ci izlazni put iz \v sume?", ako smo \v sumu i izlazni put definisali kao:
\df \v Suma je zatvoren, konveksan planaran skup.
\df Put je neprekidan i rektifiabilan planaran luk.
\df Za put $P$ ka\v zemo da je izlazni put iz \v sume $F$ ako se put $P$ ne mo\v ze postaviti unutar \v sume $P$ bez presecanja njene granice.  
\subsection[Neki poznati rezultati]{Neki poznati rezultati}
\bigskip
Re\v senje ovog problema je poznato za mali broj \v suma. Da bi se definisale neke klase \v suma za koje je problem re\v sen, potrebne su slede\' ce definicije:
\df Za ograni\v cenu \v sumu, dijametar se defini\v se kao najve\' ca udaljenost izme\dj u neke dve ta\v cke \v sume.
\df \v Sirina puta je najmanja udaljenost izme\dj u neke dve paralelne prave takve da se taj put nalazi izme\dj u njih.
\\
\bigskip
\\
Tri klase \v suma za koje je problem re\v sen su:
\begin{enumerate}
\item \textbf{"Debele" \v sume} iz kojih je najbolji izlazni put oblika du\v zi. \v Suma je "debela" ako je ograni\v cena i pokriva neki romb sa uglom od $60^\circ$ i velikom dijagonalom du\v zine svog dijametra.
\\
Neki primeri  "debelih" \v suma su kru\v znica i svi pravilni mnogouglovi ve\' ci od trougla.
\item \textbf{"Tanke" \v sume} iz kojih je najbolji izlazni put oblika Zalgaller-ove putanje, koja je najkra\' ca putanja \v sirine 1. \v Suma je "tanka" ako se u nju mo\v ze upisati Zalgaller-ov pravougaonik, tako da njegove du\v ze stranice le\v ze na granici posmatrane \v sume. Zalgaller-ov pravougaonik je pravougaonik \v cija je kra\' ca stranica 1, a du\v za du\v zine Zalgaller-ove putanje.
\\
Najpoznatiji primer "tanke" \v sume je \v suma koju obrazuju dve paralelne prave na udaljenosti 1.
\item \textbf{Neki jednakokraki trouglovi} iz kojih je najbolji izlazni put Besicovitch-eva Z kriva.
\end{enumerate}

\subsection[Uop\v stenje problema]{Uop\v stenje problema}
\bigskip
Prvo, uop\v stimo ovaj problem na vi\v sedimenzione prostore. Za to je samo potrebno predefinisati \v sumu i put tako da vi\v se ne \v zive u $\mathbb{R}^2$, ve\' c u $\mathbb{R}^n$.
\df n-dimenziona \v suma je zatvoren, konveksan skup u $\mathbb{R}^n$.
\df n-dimenzioni put je neprekidan i rektifiabilan luk u $\mathbb{R}^n$.
\df Za n-dimenzioni put ka\v zemo da je izlazan iz n-dimenzione \v sume ako ne mo\v ze biti postavljen unutar nje bez presecanja njenih granica.
\df \v Sirina n-dimenzionog puta je najmanja udaljenost izme\dj u dve paralelne hiperravni takve da se taj put nalazi izme\dj u njih.
\\
\smallskip
\\
\indent Drugo uop\v stenje koje \' ce biti posmatrano u ovom radu je problem vi\v se ljudi unutar \v sume. U kontekstu originalnog problema, pitanje bi bilo postavljeno kao:
\begin{center}
"Grupa od n ljudi je zajedno izgubljena unutar \v sume poznatog oblika i dimenzija. Koja je najkra\' ca suma puteva koje oni moraju da pre\dj u tako da znaju da \' ce barem jedan od njih iza\' ci iz \v sume?"
\end{center}
\indent Formalno, to bi se definisalo kao:
\df Put za n ljudi je skup k puteva koji kre\' cu iz iste ta\v cke.
\df Du\v zina puta za n ljudi je suma pojedina\v cnih puteva.
\df Za put za k ljudi $P_n$ ka\v zemo da je izlazan put iz \v sume $F$ ako za bilo koji po\v cetni polo\v zaj, barem jedan od n puteva iz skupa izlazi iz $F$.
\\
\smallskip
\\
Na primer, iz \v sume u $\mathbb{R}^2$ ograni\v cene sa dve prave na udaljenosti 1 je za jednu osobu najkra\' ci izlazni put Zalgaller-ova putanja, koja je du\v zine oko 2.278291644, a za tri \v coveka je jedan od izlaznih puteva cine tri du\v zi du\v zine $\frac{2}{3}$ pod uglovima od $120^\circ$. Ovaj izlazni put za 3 osobe ima du\v zinu 2, te je kra\' ci od Zalgaller-ove krive.
\\
\smallskip
\\
\indent Prirodno se postavlja slede\' ce pitanje:
\begin{center}
"Grupa ljudi je zajedno izgubljena u \v sumi poznatih dimenzija. Koliko optimalno treba da ih bude, i koliki put moraju da pre\dj u, tako da znaju da \' ce iza\' ci iz \v sume?"
\end{center}
Formalnije, ako je za \v sumu $F$ du\v zina najkra\' ceg izlaznog puta za $k$ ljudi $S_k(F)$, tra\v zi se:
$$S(F)=\lim_{n \to \infty} S_n(F)$$
Ovaj limes je definisan jer je niz $S_n(F)$ nerastu\' ci, jer u izlaznom putu za k+1 ljudi, jedan \v covek mo\v ze da stoji i stoga $S_{k+1}(F)\leqslant S_{k}(F)$. Ovaj niz je tako\dj e ograni\v cen odozdo sa nulom jer ozna\v cava sumu nekih du\v zina, te ovaj limes mora da postoji. 
\\
\section[Traka u n dimenzija]{Traka u n dimenzija}
\bigskip
 Sli\v cno problemu \v sume oblika beskona\v cne trake, tj. \v sume u ravni koja je ograni\v cena sa dve paralelne prave na rastojanju 1 \' cemo posmatrati \v sume u $\mathbb{R}^n$ ograni\v cene sa dve paralelne hiperravni na rastojanju 1. Ovako definisanu \v sumu u $\mathbb{R}^n$ zovimo \textbf{trakom u n dimenzija} i ozna\v cimo je sa $T_n$.
\\
\smallskip
\\
\indent \v Sume ovog oblika su ve\' c posmatrane u kontekstu ispitivanja \v sirina putanja, po\v sto va\v zi ekvivalencija izme\dj u toga da n-dimenzioni put ima \v sirinu barem jedan i da je izlazna za \v sumu $T_n$. Ghomi je 2018. pokazao da va\v zhi:
\teo[Ghomi] Neka je $\gamma: [a,b]\to \mathbb{R}^n$ rektifiabilan luk i neka je $L$ njena du\v zina i $\omega$ njena \v sirina. Tada va\v zi:
$$\frac{L}{\omega}\geqslant \sqrt{2.2782^2+9(n-2)}$$
\\
\indent Stoga, donje ograni\v cenje du\v zine izlaznog puta iz $T_n$ je $ \sqrt{2.2782^2+9(n-2)}$.
\\
\\
Zalgaller je konstruisao put u $\mathbb{R}^3$ jedini\v cne \v sirine i du\v zine ne ve\' ce od 3.9215, \v sto je blizu ograni\v cenja 3.7669 dobijenog Teoremom 2.1.
\\
\pred $S(T_n)=S_{n+1}(T_n)=\sqrt{n\cdot\left\lceil\frac{n}{2}\right\rceil\cdot\left( n-\left\lceil\frac{n}{2}\right\rceil+1\right)}$.
\\
\\
\indent Intuicija iza postavljanja ovog predloga je to \v sto se na primerima $\mathbb{R}^2$ i $\mathbb{R}^3$ ispostavlja da je put posmatran dalje u ovom poglavlju sa 3, odnosno 4 osobe kra\' ci od najkra\' ceg mogu\' ceg puta za jednog \v coveka. Tako\dj e je konveksni omota\v c ovih puteva ,,najmanje" telo koje postoji u posmatranom prostoru, tj. simpleks i ovaj put deluje kao najkra\' ci put \v ciji je ovo konveksni omota\v c.


\smallskip
\subsection[Minimalan simpleks]{Minimalan simpleks}
\bigskip
\indent Slede\' ce tvr\dj enje daje gornje ograni\v cenje za najmanje dimenzije pravilnog n-simpleksa koji se ne mo\v ze postaviti unutar trake u n dimenzija. Iz njega sledi gornje ograni\v cenje za izlaznu putanju iz ovakve \v sume u $\mathbb{R}^n$ za jednog \v coveka, kao i gornje ograni\v cenje tog puta za $n+1$ ljudi. 
\tvr Izme\dj u dve hiperravni u $\mathbb{R}^n$ na udaljenosti 1 se ne mo\v ze postaviti pravilan n-simpleks stranice $\sqrt{\frac{2\left\lceil \frac{n}{2}\right\rceil\cdot \left( n-\left\lceil\frac{n}{2}\right\rceil+1\right)}{n+1}}$ tako da ga ni jedna od ove dve hiperravni ne se\v ce.
\\
\smallskip
\\
\begin{proof}
Posmatrajmo u $\mathbb{R}^{n+1}$ simpleks odre\dj en temenima:
$$A_1=(0, 0, 0, ..., 0, 0)$$
$$A_2=(-1, 1, 0, ..., 0, 0)$$
$$...$$
$$A_{n+1}=(-1,0,0,..., 0, 1)$$
Ovaj simpleks, iako je definisan u $\mathbb{R}^{n+1}$ ima n dimenzija, tj. pripada podprostoru $\mathbb{R}^n$. 
 \\
\indent Po\v sto \v zelimo da na\dj emo maksimalan n-simpleks koji mo\v zemo postaviti unutar trake u n dimenzija, to je ekvivalentno tome da za fiksan n-simpleks na\dj emo minimalnu udaljenost izme\dj u dva podprostora dimenzije $n-1$ takvih da kako god da simpleks stoji izme\dj u njih, neka od njih ga sigurno se\v ce posmatrani simpleks.
\\
\indent Posmatrajmo neki $(n-1)$-dimenzioni podprostor koji sadr\v zi teme $A_1$ i pripada podprostoru generisanom ta\v ckama $A_1, A_2,..., A_{n+1}$, takav da su ta\v cke $A_2, A_3,..., A_{n+1}$ sa iste strane ovog podprostora. Neka je $\vv{x}=(x_1, x_2,..., x_{n+1})$ jedini\v cni vektor normalan na ovaj podprostor. Za njega tako\dj e va\v zi da pripada podprostoru generisanom ta\v ckama $A_1, A_2,..., A_{n+1}$, te za njega va\v zi da je linearna kombinacija vektora $\vv{A_1A_2}, \vv{A_1A_3},...,\vv{A_1 A_{n+1}}$.
$$\lambda_2 \cdot \vv{A_1A_2}+\lambda_3 \cdot \vv{A_1A_3}+...+\lambda_{n+1}\cdot \vv{A_1A_{n+1}}=\vv{x}$$
Iz ovoga sledi da va\v zi sistem jedna\v cina:
$$ -\lambda_2-\lambda_3-...-\lambda_{n+1}=x_1$$
$$\lambda_2=x_2$$
$$\lambda_3=x_3$$
$$...$$
$$\lambda_{n+1}=x_{n+1}$$
$$\Rightarrow x_1+x_2+...+x_{n+1}=0 $$
Tako\dj e, po\v sto je $\vv{x}$ jedini\v can vektor normalan na posmatrani $(n-1)$-dimenzioni podprostor, za svako $i\in {2, 3,..., n+1}$, skalarni proizvod $\vv{A_1A_i}\cdot \vv{x}$ predstavlja rastojanje izme\dj u temena $A_i$ i posmatranog $(n-1)$-dimenzionog podprostora. Neka su ova rastojanja redom $l_2, l_3,..., l_{n+1}$.
$$l_2=(-1, 1, 0,..., 0)\cdot (x_1, x_2, x_3,..., x_{n+1})=x_2-x_1$$
$$l_3=x_3-x_1$$
$$...$$
$$l_{n+1}=x_{n+1}-x_1$$
Po\v sto su $A_2, A_3,..., A_{n+1}$ sa iste strane podprostora na koji je normalan vektor $\vv{x}$, ovi skalarni proizvodi su svi istog znaka. Bez umanjenja op\v steg, recimo da su nenegativni.
Predstavimo $x_1, x_2,..., x_{n+1}$ u funkciji od $l_2, l_3,..., l_{n+1}$:
$$x_1=-\frac{l_2+l_3+...+l_{n+1}}{n+1}$$
$$x_2=\frac{n\cdot l_2-l_3-l_4-...-l_{n+1}}{n+1}$$
$$x_3=\frac{-l_2+n\cdot l_3-l_4-...-l_{n+1}}{n+1}$$
$$...$$
$$x_{n+1}=\frac{-l_2-l_3-l_4-...+n\cdot l_{n+1}}{n+1}$$
\indent Po\v sto za vektor $\vv{x}$ znamo da je jedini\v can, va\v zi:
$$x_1^2+ x_2^2+...+x_{n+1}^2=1$$
$$\Rightarrow \left(\frac{-l_2-l_3-...-l_{n+1}}{n+1}\right)^2+\left(\frac{n\cdot l_2-l_3-...-l_{n+1}}{n+1}\right)^2+...+\left(\frac{-l_2-l_3-...+n\cdot l_{n+1}}{n+1}\right)^2=1$$
$$\Rightarrow \frac{1}{(n+1)^2}\left[ (n^2+n)\cdot {\sum_{i=2}^{n+1} {l_i^2}} - (2n+2)\cdot {\sum_{i=2, i<j}^{n+1} {l_i\cdot l_j} } \right]=1$$
$$\Rightarrow n\cdot {\sum_{i=2}^{n+1} {l_i^2}} - 2\cdot {\sum_{i=2,  i<j}^{n+1} {l_i\cdot l_j} } =n+1$$
\\
\indent Problem koji poku\v avamo da re\v simo je da na\dj emo najve\' cu udaljenost na koju mo\v zemo da postavimo drugi $(n-1)$-dimenzioni podprostor paralelan posmatranom, tako da on sigurno se\v ce posmatrani simpleks. Primetimo da je tra\v zena udaljenost ta\v cno minimum po svim posmatranim $(n-1)$-dimenzionim podprostorima od $max\{l_2, l_3,..., l_{n+1}\}$.
\\
\\
\indent U ovom izrazu figuri\v su samo $l_2, l_3,..., l_{n+1}$, te iz njega \v zelimo da pronadjemo ovu vrednost. Potrebno nam je bilo koje ogrni\v cenje za ovu vrednost, ali smatramo da se optimalna vrednost dobija kada je polovina ovih udaljenosti 0, tj. polovina temena pripada posmatranom $(n-1)$-dimenzionom podprostoru, a druga polovina je na jednakoj udaljenosti od nje, tj. $l_2=l_3=...=l_{\lceil \frac{n}{2}\rceil+1}=l$. Time se dobija:
$$n\cdot \left \lceil{\frac{n}{2}}\right \rceil\cdot l^2-2\cdot {{\lceil{\frac{n}{2}}\rceil}\choose {2}}\cdot l^2=n+1$$
$$\left( n\cdot \left \lceil \frac{n}{2} \right \rceil - {\left \lceil \frac{n}{2} \right\rceil} \cdot \left( \left \lceil \frac{n}{2}\right \rceil -1 \right) \right) \cdot l^2=n+1$$
$$\left \lceil \frac{n}{2} \right \rceil \cdot \left( n-\left \lceil \frac{n}{2}\right \rceil +1 \right)\cdot l^2=n+1$$
$$\Rightarrow l=\sqrt{\frac{n+1}{\left\lceil \frac{n}{2}\right\rceil \cdot \left( n-\left\lceil \frac{n}{2}\right\rceil+1\right)}}$$
Po\v sto je stranica ovog simpleksa du\v zine $\sqrt2$, odnos stranice simpleksa i udaljenosti dva $(n-1)$-dimenziona podprostora izme\dj u  kojih se taj tetraedar ne mo\v ze postaviti izme\dj u njih je:
$$\frac{\sqrt2}{l}=\sqrt{\frac{2\left\lceil \frac{n}{2}\right\rceil\cdot \left( n-\left\lceil\frac{n}{2}\right\rceil+1\right)}{n+1}}$$
Stoga, izme\dj u dva paralelna $(n-1)$-dimenziona podprostora na udaljenosti $1$ ne mo\v ze da se postavi n-simpleks stranice  $\sqrt{\frac{2\left\lceil \frac{n}{2}\right\rceil\cdot \left( n-\left\lceil\frac{n}{2}\right\rceil+1\right)}{n+1}}$.
$\square$
\end{proof}
\\
\smallskip
\\
\subsection[Jedan \v covek]{Jedan \v covek}
\bigskip
\teo Put koji formiraju po trojkama nekoplanarne ivice $n-simpleksa$ \v cije su ivice du\v zine $\sqrt{\frac{2\left\lceil \frac{n}{2}\right\rceil\cdot \left( n-\left\lceil\frac{n}{2}\right\rceil+1\right)}{n+1}}$ je izlazni put iz trake u n dimenzija i taj put ima du\v zinu:
$$L_{n,1}=n\cdot \sqrt{\frac{2\left\lceil \frac{n}{2}\right\rceil\cdot \left( n-\left\lceil\frac{n}{2}\right\rceil+1\right)}{n+1}}.$$
\\
\smallskip
\begin{proof}
Po Teoremi 2.1, ovaj simpleks se ne mo\v ze postaviti izme\dj u posmatranih podprostora, tj. barem jedna od granica \v sume se\v ce simpleks, kako god ga postavili u \v sumi.
\\
\indent Po\v sto postoji put po nekoplanarnim ivicama simpleksa koji obilazi sva temena i on je neprekidan i konveksni omota\v c tog puta je ba\v s ovaj simpleks. Zato \v sto posmatrani simpleks izlazi iz ove \v sume, tj. jedan od ova dva $(n-1)$-dimenziona podprostora ga se\v ce, on mora da se\v ce i put \v ciji je taj simpleks konveksni omota\v c. Stoga, ova putanja je izlazna i ima du\v zinu:
$$L_{n,1}=n\cdot \sqrt{\frac{2\left\lceil \frac{n}{2}\right\rceil\cdot \left( n-\left\lceil\frac{n}{2}\right\rceil+1\right)}{n+1}}.$$
$\square$
\end{proof}
\\
\\
Stoga, va\v zi slede\' ce:
$$S_1(T_n)\leqslant L_{n, 1}=n\cdot \sqrt{\frac{2\left\lceil \frac{n}{2}\right\rceil\cdot \left( n-\left\lceil\frac{n}{2}\right\rceil+1\right)}{n+1}}$$
\\
\smallskip
\subsection[n+1 \v covek]{n+1 \v covek} 
\bigskip
\teo Put koji formira $n+1$ du\v zi od centra do svih temena pravilnog $n$-simpleksa  \v cije su ivice du\v zine $\sqrt{\frac{2\left\lceil \frac{n}{2}\right\rceil\cdot \left( n-\left\lceil\frac{n}{2}\right\rceil+1\right)}{n+1}}$ je izlazni put za $n+1$ ljudi iz trake u n dimenzija i taj put ima du\v zinu:

$$L_{n,n+1}=\sqrt{n\cdot\left\lceil\frac{n}{2}\right\rceil\cdot\left( n-\left\lceil\frac{n}{2}\right\rceil+1\right)}$$.
\\
\smallskip
\begin{proof}
Sli\v cno dokazu Teoreme 2.2, po\v sto svaki od $n+1$ ljudi ide od centra posmatranog simpleksa do nekog od njegovih temena, tako da do svakog temena ide ta\v cno jedan \v covek, konveksni omota\v c skupa ovih puteva je ba\v s taj simpleks. Po\v sto barem jedan od posmatranih $(n-1)$- dimenzionih podprostora se\v ce konveksni omota\v c ovog puta, barem jedan od njih mora se\' ci i taj put. 
\\
 Odnos rastojanja izme\dj u centra simpleksa i temena sa du\v zinom ivice je:
$$\frac{\sqrt{\left( 1-\frac{1}{n+1}\right)^2+n\cdot{\frac{1}{n+1}}^2}}{\sqrt2}=\frac{\sqrt{n^2+n}}{\sqrt2 \cdot(n+1)}$$
\indent Onda je rastojanje izme\dj u centra i temena u simpleksu stranice  $\sqrt{\frac{2\left\lceil \frac{n}{2}\right\rceil\cdot \left( n-\left\lceil\frac{n}{2}\right\rceil+1\right)}{n+1}}$ je:
$$L_{n,n+1}=\frac{1}{n+1}\cdot\sqrt{n\cdot\left\lceil\frac{n}{2}\right\rceil\cdot\left( n-\left\lceil\frac{n}{2}\right\rceil+1\right)}$$
Po\v sto svaki od $n+1$ ljudi koji kre\' cu iz centra ovog simpleksa ima put du\v zine $S$, ukupna du\v zina puta je:
$$L_{n,n+1}=\sqrt{n\cdot\left\lceil\frac{n}{2}\right\rceil\cdot\left( n-\left\lceil\frac{n}{2}\right\rceil +1\right)}$$
$\square$
\end{proof}
\\
\\
Stoga, va\v zi slede\' ce:
$$S_{n+1}(T_n)\leqslant L_{n,n+1}= \sqrt{n\cdot\left\lceil\frac{n}{2}\right\rceil\cdot\left( n-\left\lceil\frac{n}{2}\right\rceil +1\right)}$$

\section[k-traka u n dimenzija]{k-traka u n dimenzija}
\bigskip
\indent Traka u n dimenzija predstavlja upo\v stenje klasi\v cnog primera \v sume u $\mathbb{R}^2$ ograni\v cene sa dve paralelne prave na udaljenosti 1. Ovo je uop\v stenje koje nastaje interpretacijom ove \v sume kao prostora ograni\v cenog hiperravnima u nekom prostoru i u slu\v caju $\mathbb{R}^2$ je to jedina takva \v suma koju ima smisla posmatrati. U prostorima vi\v sih dimenzija od $\mathbb{R}^2$ to nije slu\v caj.
\\
\smallskip
\\
\indent Defini\v simo \textbf{k-traku u n dimenzija} za svako prirodno n i za svako $1\leqslant k\leqslant n-1$  kao \v sumu odredjenu svim ta\v ckama iz $\mathbb{R}^n$ na udaljenosti manjoj ili jednakoj $\frac{1}{2}$ od nekog k-dimenzionog podprostora $\mathbb{R}^n$. 
\\
\teo Najkra\' ca izlazna putanja iz k-trake u $\mathbb{R}^n$ je ista kao najkra\' ca izlazna putanja iz trake u $\mathbb{R}^{k+1}$.
\\
\smallskip
\\
\begin{proof}
Posmatrajmo presek nekog podprostora dimenzije k+1 i k-trake. Taj presek \' ce o\v cigledno pripadati projekciji k-trake na ovaj podprostor.
\\
\indent  Svaka ta\v cka k-trake u $\mathbb{R}^n$ se mo\v ze predstaviti kao zbir neke ta\v cke koja pripada unutra\v snjosti sfere jedini\v cnog pre\v cnika $\mathbb{S}^{n-k}$ i nekog vektora iz linearno nezavisnog podprostora $\mathbb{R}^k$. To zna\v ci da \' ce projekcija k-trake na bilo koji podprostor dimenzije k+1 imati jedini\v cnu \v sirinu. Ovo zna\v ci da je izlazna putanja iz trake u  $\mathbb{R}^{k+1}$ dovoljna. 
\\
\indent Ako uzmemo podprostor dimenzije k+1 \v generisan istim k podprostorom kojim je generisana traka, i proizvoljnim vektorom sa sfere, projekcija k-trake na njega bi\' ce ba\v s traka u podprostoru $\mathbb{R}^{k+1}$ , \v cime je dokaz kompletan.
$\square$
\end{proof}




 
\begin{thebibliography}{99}


\bibitem{Ward}
    J. W. Ward,
    \textit{Eploring the Bellman Forest Problem},
    Spring (2008)

\bibitem{Finch}
	S. R. Finch, J. E. Wetzel,
	\textit{Lost in a forest},
	The American Mathematical Monthly, 111(8) : 645-654 (2004)

\bibitem{Ghomi}
 	M. Ghomi,
	\textit{The length, width, and inradius of space curves},
	arXiv preprint(2018), arXiv: 1605.01144v3

\bibitem{Zalgaller}
	V. A. Zalgaller,
	\textit{ The problem of the shortest space curve of unit width},
	 Mat. Fiz. Anal. Geom,1(3-4):454–461 (1994)

\bibitem{Gibbs}
	P. Gibbs, 
	\textit{Bellman’s Escape Problem for Convex Polygons},
	viXra:1606.0050 (2016)

\bibitem{Gibbs2}
	P. Gibbs,
	\textit{Lost in a Isosceles Triangle},
	viXra:1606.0015 (2016)

\bibitem{Wetzel}
	Y. Movshovich, J. E. Wetzel,
	\textit{Drapeable unit arcs fit in the unit $30^\circ$ sector},
	 Adv. Geom. 17: 497–506 (2017)


\end{thebibliography}


\end{document}
