\documentclass[11pt,letter]{article}
\usepackage{amsfonts,epsf,amsmath,amssymb,graphicx,tabularx,booktabs}
\usepackage{graphicx}
\usepackage{amsmath}
\usepackage{amsfonts}
\usepackage{amssymb}
\usepackage{booktabs}
\usepackage{tabularx}
\usepackage{epstopdf}
\usepackage[ruled,linesnumbered]{algorithm2e}
\usepackage{subfig}
\usepackage{multicol}
\usepackage{enumerate}
\usepackage{url}
\usepackage{float}
\usepackage{indentfirst}
\usepackage{babel}[serbian]
\usepackage{esvect}


\newtheorem{teo}{\bf Teorema}[section]
\newtheorem{posl}[teo]{\bf Posledica}
\newtheorem{lem}[teo]{\bf Lema}
\newtheorem{pred}[teo]{\bf Hipoteza}
\newtheorem{tvr}[teo]{\bf Tvr\dj enje}
\newtheorem{pro}[teo]{\bf Problem}
\newtheorem{df}[teo]{\bf Definicija}

\newcommand{\comment}[1]{\ $[\![${\normalsize #1}$]\!]$ \ }
\newcommand{\proof}{\noindent{\bf Proof.\ }}
\newcommand{\qed}{\hfill $\square$ \bigskip}
\def\cp{\,\square\,}



\renewcommand{\tablename}{Tabela}
\renewcommand{\figurename}{Slika}
\renewcommand{\refname}{Literatura}
\renewcommand{\abstractname}{Apstrakt}
\renewcommand{\proof}{\textbf{Dokaz.}}

% Centering the EPS graphics inside the FIGURE environment
\def\slika #1{\begin{center} \epsffile{#1} \end{center}}

\newenvironment{wlist}
{\vspace{-10pt}
\begin{list}{}
{\setlength{\labelwidth}{15mm} \setlength{\partopsep}{0pt} \setlength{\parskip}{0pt}
\setlength{\topsep}{0pt} \setlength{\itemsep}{0pt} \setlength{\parsep}{0pt}
\setlength{\labelsep}{10pt} \setlength{\leftmargin}{15mm}}
\item[]} {\end{list} \vspace{2pt}
\smallskip}

\textwidth 16cm \textheight 23.5cm \oddsidemargin 0.0cm \evensidemargin 0.0cm \voffset -2.2cm


\renewcommand{\thefootnote}{\fnsymbol{footnote}}
\begin{document}

\title{\bf{Problem izlaska iz \v sume}}

\author{
Katarina Krivoku\'ca \\
Matemati\v cka gimnazija, Beograd, Srbija\\
e-mail: \tt{katarinakrivokuca@outlook.com} \\
\and
Dimitrije Gluk\v cevi\' c \\
Gimnazija ,,Svetozar Markovi\' c", Ni\v s, Srbija \\
e-mail: \tt{dimchee90@gmail.com} \\
}

\date{\today}
\maketitle %\vspace{-0.5cm}

\begin{abstract}
Belmanov problem tra\v zenja najkra\' ceg izlaznog puta iz \v sume poznatog oblika i dimenzija je re\v sen za mali broj \v suma. Jedan od re\v senih slu\v cajeva je traka ograni\v cena sa dve paralelne prave na jedini\v cnoj udaljenosti. Ta\v cnije, Zalgaler \cite{Zalgaller} je prona\v sao najkra\' ci put jedini\v cne \v sirine i pokazano je da je to najkra\' ci izlazni put iz posmatrane \v sume. Tako\dj e su istra\v zivani i najkra\' ci putevi \cite{Zalgaller3d} jedini\v cne \v sirine u vi\v sim dimenzijama. Gomi je \cite{Ghomi} poo\v strio gornje ograni\v cenje za du\v zinu takvog puta u prostoru i prona\v sao donje ograni\v cenje za du\v zinu takvog puta u $\mathbb{R}^n$. U ovom radu \' cemo predstaviti jedno gornje ograni\v cenje za du\v zinu najkra\' ceg puta jedini\v cne \v sirine u $\mathbb{R}^n$, kao i modifikaciju Belmanovog problema u kojoj vi\v se ljudi kre\' ce iz iste ta\v cke i smatra se da su iza\v sli iz \v sume ako barem jedan od njih iza\dj e, pri \v cemu se za du\v zinu puta smatra suma du\v zina pojedina\v cnih puteva. Dalje je pokazano jedno gornje i jedno donje ograni\v cenje za du\v zinu najkra\' ceg puta jedini\v cne \v sirine u $\mathbb{R}^n$ pri posmatranoj modifikaciji problema, kao i dokaz optimalnosti jedne modifikovane putanje u ravni pri restrikciji konveksnog omota\v ca puta na trougao. Dosada\v snji rad i intuicija ukazuju na to da je pri posmatranoj modifikaciji u $\mathbb{R}^n$ najkra\' ci put jedini\v cne \v sirine ba\v s put za $(n+1)$-nog \v coveka konstruisan u ovom radu.


\end{abstract}

\section[Uvod]{Uvod}
\bigskip
\subsection[Originalni problem]{Originalni problem}
\bigskip
Originalni problem ,,Izgubljeni u \v sumi" je postavio Belman 1956.  \cite{Bellman} i on je glasio:
\begin{center}
\textit{,,Koji je najkra\' ci put kojim \v covek izgubljen u \v sumi poznatog oblika i dimenzija treba da ide tako da posle tog puta zna da \' ce sigurno iza\' ci iz nje?"}
\end{center}

\indent Formalnije, problem se defini\v se kao ,,Koji je najkra\' ci izlazni put iz \v sume?", ako smo \v sumu i izlazni put definisali kao:
\begin{df} \v Suma je zatvoren, konveksan planaran skup. \end{df}
\begin{df} Put je neprekidna i rektifiabilna kriva.\end{df}
\begin{df} Za put $P$ ka\v zemo da je izlazni put iz \v sume $F$ ako se put $P$ ne mo\v ze postaviti unutar \v sume $P$ bez presecanja njenog ruba.  \end{df}
\subsection[Neki poznati rezultati]{Neki poznati rezultati}
\bigskip
Re\v senje ovog problema je poznato za mali broj \v suma. Da bi se definisale neke klase \v suma za koje je problem re\v sen, potrebne su slede\' ce definicije:
\begin{df} Za ograni\v cenu \v sumu, dijametar se defini\v se kao najve\' ca udaljenost izme\dj u neke dve ta\v cke \v sume.\end{df}
\begin{df} \v Sirina puta je najmanja udaljenost izme\dj u neke dve paralelne prave takve da se taj put nalazi izme\dj u njih. \end{df}

Tri klase \v suma za koje je problem re\v sen su:
\begin{enumerate}
\item \textbf{,,Debele" \v sume} \cite{Finch} su ograni\v cene i pokrivaju neki romb sa uglom od $60^\circ$ i velikom dijagonalom du\v zine dijametra \v sume. Najkra\' ci izlazni put iz njih je du\v z \v cija je du\v zina jednaka dijametru posmatrane \v sume. 
\\
Neki primeri  ,,debelih" \v suma su kru\v znica i svi pravilni mnogouglovi sa vi\v se od 3 stranice.
\item \textbf{,,Tanke" \v sume} \cite{Zalgaller} su one u koje se mo\v ze upisati tzv. Zalgalerov pravougaonik, tako da njegove du\v ze stranice le\v ze na rubu posmatrane \v sume. Zalgalerov pravougaonik je pravougaonik \v cija je kra\' ca stranica 1, a du\v za du\v zine Zalgalerove putanje, koja je tako\dj e najkra\' ca izlazna putanja iz \v suma ove klase.
\\
Najpoznatiji primer ,,tanke" \v sume je \v suma koju obrazuju dve paralelne prave na udaljenosti 1.
\item \textbf{Neki jednakokraki trouglovi} \cite{Besikovic}, tj. jednakokraki trouglovi \v ciji uglovi pripadaju nekom konkretnom rasponu, iz kojih je najbolji izlazni put Besikovi\v ceva ,,Z" kriva.
\end{enumerate}
\smallskip
\subsection[Uop\v stenje problema]{Uop\v stenje problema}
\bigskip
Prvo, uop\v stimo ovaj problem na vi\v sedimenzione prostore. Za to je potrebno predefinisati \v sumu i put tako da vi\v se ne ,,\v zive" u $\mathbb{R}^2$, ve\' c u $\mathbb{R}^n$.
\smallskip
\begin{df} $n$-dimenziona \v suma je zatvoren, konveksan skup u $\mathbb{R}^n$.\end{df}
\begin{df} $n$-dimenzioni put je neprekidna i rektifiabilna kriva u $\mathbb{R}^n$.\end{df}
\begin{df} Za $n$-dimenzioni put ka\v zemo da je izlazan iz $n$-dimenzione \v sume ako ne mo\v ze biti postavljen unutar nje bez presecanja njenog ruba.\end{df}
\begin{df} \v Sirina $n$-dimenzionog puta je najmanja udaljenost izme\dj u dve paralelne hiperravni takve da se taj put nalazi izme\dj u njih. \end{df}
\smallskip

\indent Drugo uop\v stenje koje \' ce biti posmatrano u ovom radu je problem vi\v se ljudi unutar \v sume. U kontekstu originalnog problema, pitanje bi bilo postavljeno kao:
\begin{center}
\textit{"Grupa od $k$ ljudi je zajedno izgubljena unutar \v sume poznatog oblika i dimenzija. Koja je najmanja suma du\v zina puteva koje oni moraju da pre\dj u tako da znaju da \' ce barem jedan od njih iza\' ci iz \v sume?"}
\end{center}
\newpage
Formalno, to bi se definisalo kao:
\begin{df} Put za $k$ ljudi je skup $k$ puteva koji kre\' cu iz iste ta\v cke. \end{df}
\begin{df} Du\v zina puta za $k$ ljudi je suma du\v zina pojedina\v cnih puteva.\end{df}
\begin{df} Za put za $k$ ljudi ka\v zemo da je izlazan put iz \v sume $F$ ako za bilo koji po\v cetni polo\v zaj, barem jedan od $k$ puteva iz skupa izlazi iz $F$.\end{df}
\smallskip

\indent Na primer, iz \v sume u $\mathbb{R}^2$ ograni\v cene sa dve prave na jedini\v cnoj udaljenosti, za jednu osobu je najkra\' ci izlazni put Zalgalerova putanja, koja je du\v zine oko 2.278291644, a za tri \v coveka jedan od izlaznih puteva \v cine tri du\v zi du\v zine $\frac{2}{3}$ pod uglovima od $120^\circ$. Ovaj izlazni put za 3 osobe ima du\v zinu 2, te je kra\' ci od Zalgalerove krive.
\\
\smallskip
\\
\indent Prirodno se postavlja slede\' ce pitanje:
\begin{center}
\textit{,,Grupa ljudi je zajedno izgubljena u \v sumi poznatih dimenzija. Pretpostavljaju\' ci da je grupa dovoljno velika, koji je njihov najkra\' ci izlazan put iz \v sume?"}
\end{center}
Formalnije, ako je za \v sumu $F$ du\v zina najkra\' ceg izlaznog puta za $k$ ljudi $S_k(F)$, tra\v zi se:
$$S(F)=\lim_{k \to \infty} S_k(F)$$

\indent Ovaj limes je definisan jer je niz $S_k(F)$ nerastu\' ci, jer u izlaznom putu za $k+1$ ljudi, jedan \v covek mo\v ze da stoji i stoga $S_{k+1}(F)\leqslant S_{k}(F)$. Ovaj niz je tako\dj e ograni\v cen odozdo nulom jer predstavlja sumu nekih du\v zina, te ovaj limes mora da postoji. 
\\
\subsection[Cilj rada]{Cilj rada}
\bigskip
U ovom radu posmatramo uop\v stenje Belmanovog problema u $\mathbb{R}^n$, u originalu, kao i pri modifikaciji za vi\v se ljudi navedenoj u prethodnom odeljku.
\\
\\
\indent U odeljku 2.1 je pokazano koja je najmanja du\v zina stranice pravilnog simpleksa u $\mathbb{R}^n$ koji se ne mo\v ze smestiti izme\dj u dve hiperravni na jedini\v cnoj udaljenosti. Zatim su u odeljcima 2.2 i 2.3  uo\v cene ,,\v setnje" po tom simpleksu koje su izlazne putanje iz \v sume ograni\v cene sa dve paralelne hiperravni na jedini\v cnoj udaljenosti u $\mathbb{R}^n$ redom za jednog i $(n+1)$-nog \v coveka, \v cime su data dva gornja ograni\v cenja za du\v zinu izlaznog puta iz ovih \v suma - prvo va\v zi za uop\v stenje originalnog problema, a drugo (koje je ja\v ce od prvog) va\v zi za modifikaciju problema. 
\\
\\
\indent Nadalje se sve posmatra pri modifikovanoj verziji problema.
\\
\\
\indent Tre\' ce poglavlje se bavi donjim ograni\v cenjem za du\v zinu najkra\' ceg izlaznog puta iz \v suma posmatranih u poglavlju 2, u zavisnosti od toga koliko ljudi poku\v sava da iza\dj e iz \v sume.
\\
\\
\indent U \v cetvrtom poglavlju su posmatrane \v sume koje su uo\v cene kao drugi na\v cin da se u $\mathbb{R}^n$ uop\v sti planarna \v suma ograni\v cena sa dve paralelne prave na jedini\v cnoj udaljenosti. Pokazano je da je problem izlaska iz svake od tih \v suma ekvivalentan problemu izlaska iz \v sume ograni\v cene sa dve paralelne hiperravni na jedini\v cnoj udaljenosti u prostoru neke druge (konkretno odre\dj ene) dimenzije.
\\
\\
\indent U petom poglavlju je posmatran problem izlaska iz planarne \v sume ograni\v cene sa dve paralelne prave na jedini\v cnoj udaljenosti. Pokazano je da je, pri restrikciji konveksnog omota\v ca puta na trougao, najkra\' ci put sa ma koliko ljudi ba\v s konstruisani put sa tri osobe du\v zine 2.
\\
\section[Traka u $\textbf{n}$ dimenzija]{Traka u $\textbf{n}$ dimenzija}
\bigskip
 Sli\v cno problemu \v sume oblika beskona\v cne trake, posmatra\' cemo \v sume u $\mathbb{R}^n$ ograni\v cene sa dve paralelne hiperravni na rastojanju 1. Ovako definisanu \v sumu u $\mathbb{R}^n$ zovimo \textbf{trakom u $\textbf{n}$ dimenzija} i ozna\v cimo je sa $T_n$.
\\
\\
\indent \v Sume ovog oblika su ve\' c posmatrane u kontekstu ispitivanja \v sirina putanja, po\v sto va\v zi ekvivalencija izme\dj u toga da $n$-dimenzioni put ima \v sirinu barem jedan i da je izlazna za \v sumu $T_n$. Gomi \cite{Ghomi} je 2018. pokazao da va\v zi:
\smallskip
\begin{teo}[Gomi] 
Neka je $\gamma: [a,b]\to \mathbb{R}^n$ rektifiabilna kriva i neka je $L$ njegova du\v zina i $\omega$ njegova \v sirina. Tada va\v zi:
$$\frac{L}{\omega}\geqslant \sqrt{2.2782^2+9(n-2)}.$$
Dalje, ako je $\gamma$ zatvoren, va\v zi:
$$\frac{L}{\omega}\geqslant \sqrt{\pi^2+16(n-2)}.$$
\end{teo}
\bigskip

\indent Stoga, donje ograni\v cenje du\v zine izlaznog puta za jednog \v coveka iz $T_n$ je $ \sqrt{2.2782^2+9(n-2)}$.
\\
\\
Zalgaler \cite{Zalgaller3d} je konstruisao put u $\mathbb{R}^3$ jedini\v cne \v sirine i du\v zine ne ve\' ce od 3.9215, \v sto je blizu ograni\v cenja 3.7669 dobijenog Teoremom 2.1.
\\
\begin{pred} $S(T_n)=S_{n+1}(T_n)=\sqrt{n\cdot\left\lceil\frac{n}{2}\right\rceil\cdot\left( n-\left\lceil\frac{n}{2}\right\rceil+1\right)}$.\end{pred}
\smallskip

\indent Intuicija iza postavljanja ovog predloga je to \v sto se na primerima $\mathbb{R}^2$ i $\mathbb{R}^3$ ispostavlja da je put posmatran dalje u ovom poglavlju sa 3, odnosno 4 osobe kra\' ci od najkra\' ceg mogu\' ceg puta za jednog \v coveka. Tako\dj e, konveksni omota\v c posmatranih puteva je simpleks u odgovaraju\' coj dimenziji, \v sto je telo sa najmanje temena koje ,,\v zivi" u posmatranom prostoru i posmatrani put iz centra simpleksa smatramo da \' ce se ispostaviti za najkra\' ci put \v ciji je ovo konveksni omota\v c, ma koliko ljudi da ga obrazuje.


\smallskip
\subsection[Minimalan simpleks]{Minimalan simpleks}
\bigskip

\indent Slede\' ce tvr\dj enje daje minimalnu du\v zinu stranice pravilnog $n$-simpleksa koji se ne mo\v ze postaviti unutar trake u $n$ dimenzija. Iz njega sledi gornje ograni\v cenje za izlaznu putanju iz ovakve \v sume u $\mathbb{R}^n$ za jednog \v coveka, kao i gornje ograni\v cenje tog puta za $n+1$ ljudi. 
\smallskip

\begin{tvr} Izme\dj u dve hiperravni u $\mathbb{R}^n$ na udaljenosti 1 se ne mo\v ze postaviti pravilan $n$-simpleks stranice $\sqrt{\frac{2\left\lceil \frac{n+1}{2}\right\rceil\cdot \left( n-\left\lceil\frac{n+1}{2}\right\rceil+1\right)}{n+1}}$ tako da ga nijedna od ove dve hiperravni ne se\v ce.
\end{tvr}

\begin{proof}
Posmatrajmo u $\mathbb{R}^{n+1}$ simpleks odre\dj en temenima:
$$A_1=(0, 0, 0, ..., 0, 0)$$
$$A_2=(-1, 1, 0, ..., 0, 0)$$
$$...$$
$$A_{n+1}=(-1,0,0,..., 0, 1)$$
Ovaj simpleks, iako je definisan u $\mathbb{R}^{n+1}$ ima $n$ dimenzija, tj. pripada potprostoru $\mathbb{R}^n$. 
 \\
\\
\indent Po\v sto \v zelimo da na\dj emo maksimalan $n$-simpleks koji mo\v zemo postaviti unutar trake u $n$ dimenzija, to je ekvivalentno tome da za fiksan $n$-simpleks na\dj emo minimalnu udaljenost izme\dj u dva paralelna potprostora dimenzije $n-1$ izme\dj u kojih se on ne mo\v ze smestiti bez presecanja.
\\
\\
\indent Posmatrajmo neki $(n-1)$-dimenzioni potprostor koji sadr\v zi teme $A_1$ i pripada potprostoru generisanom ta\v ckama $A_1, A_2,..., A_{n+1}$, takav da su ta\v cke $A_2, A_3,..., A_{n+1}$ sa iste strane ovog potprostora. Neka je $\vv{x}=(x_1, x_2,..., x_{n+1})$ jedini\v cni vektor normalan na ovaj potprostor i koji pripada potprostoru generisanom ta\v ckama $A_1, A_2,..., A_{n+1}$. Iz tog uslova, vektor $x$ mo\v zemo predstaviti kao linearnu kombinacija vektora $\vv{A_1A_2}, \vv{A_1A_3},...,\vv{A_1 A_{n+1}}$.
\\
$$\lambda_2 \cdot \vv{A_1A_2}+\lambda_3 \cdot \vv{A_1A_3}+...+\lambda_{n+1}\cdot \vv{A_1A_{n+1}}=\vv{x}$$
\\
Iz ovoga sledi da va\v zi sistem jedna\v cina:
$$ -\lambda_2-\lambda_3-...-\lambda_{n+1}=x_1$$
$$\lambda_2=x_2$$
$$\lambda_3=x_3$$
$$...$$
$$\lambda_{n+1}=x_{n+1}$$
$$\Rightarrow x_1+x_2+...+x_{n+1}=0 $$
\\
Tako\dj e, po\v sto je $\vv{x}$ jedini\v can vektor normalan na posmatrani $(n-1)$-dimenzioni potprostor, za svako $i\in {2, 3,..., n+1}$, skalarni proizvod $\vv{A_1A_i}\cdot \vv{x}$ predstavlja rastojanje izme\dj u temena $A_i$ i posmatranog $(n-1)$-dimenzionog potprostora. Neka su ova rastojanja redom $l_2, l_3,..., l_{n+1}$.
$$l_2=(-1, 1, 0,..., 0)\cdot (x_1, x_2, x_3,..., x_{n+1})=x_2-x_1$$
$$l_3=x_3-x_1$$
$$...$$
$$l_{n+1}=x_{n+1}-x_1$$
Po\v sto su $A_2, A_3,..., A_{n+1}$ sa iste strane potprostora na koji je normalan vektor $\vv{x}$, ovi skalarni proizvodi su svi istog znaka. Bez umanjenja op\v steg, recimo da su nenegativni.
Predstavimo $x_1, x_2,..., x_{n+1}$ u funkciji od $l_2, l_3,..., l_{n+1}$:
$$x_1=-\frac{l_2+l_3+...+l_{n+1}}{n+1}$$
$$x_2=\frac{n\cdot l_2-l_3-l_4-...-l_{n+1}}{n+1}$$
$$x_3=\frac{-l_2+n\cdot l_3-l_4-...-l_{n+1}}{n+1}$$
\vspace{3pt}
$$...$$
\vspace{3pt}
$$x_{n+1}=\frac{-l_2-l_3-l_4-...+n\cdot l_{n+1}}{n+1}$$
\\
\indent Po\v sto za vektor $\vv{x}$ znamo da je jedini\v can, va\v zi:
$$x_1^2+ x_2^2+...+x_{n+1}^2=1$$
$$\Rightarrow \left(\frac{-l_2-l_3-...-l_{n+1}}{n+1}\right)^2+\left(\frac{n\cdot l_2-l_3-...-l_{n+1}}{n+1}\right)^2+...+\left(\frac{-l_2-l_3-...+n\cdot l_{n+1}}{n+1}\right)^2=1$$
$$\Rightarrow \frac{1}{(n+1)^2}\left[ (n^2+n)\cdot {\sum_{i=2}^{n+1} {l_i^2}} - (2n+2)\cdot {\sum_{i=2, i<j}^{n+1} {l_i\cdot l_j} } \right]=1$$
$$\Rightarrow n\cdot {\sum_{i=2}^{n+1} {l_i^2}} - 2\cdot {\sum_{\substack{ i=2\\ i<j}}^{n+1} {l_i\cdot l_j} } =n+1$$
\\





\indent Problem je ovime sveden na nala\v zenje najve\' ce udaljenosi na koju mo\v zemo da postavimo drugi $(n-1)$-dimenzioni potprostor paralelan posmatranom, tako da on sigurno se\v ce posmatrani simpleks. Primetimo da je tra\v zena udaljenost jednaka minimumu po svim posmatranim $(n-1)$-dimenzionim potprostorima od $\max{\{l_2, l_3,..., l_{n+1}\}}$.
\\
\\
\indent Ekvivalentan problem tome bi bio, za neko fiksno $l\in \mathbb{R}^+$ maksimizovati izraz:
$$n\cdot {\sum_{i=2}^{n+1} {l_i^2}} - 2\cdot {\sum_{\substack{ i=2\\ i<j}}^{n+1} {l_i\cdot l_j} }, $$
pri uslovu $l_2,l_3,..., l_{n+1}\in [0,l]$. (Zahvaljujemo Pavlu Martinovi\' cu na pomo\' ci oko ovog dela dokaza.)
\\
\\
\indent Fiksiranjem $l_2, l_3,..., l_n$ dobijamo kvadratnu funkciju po $l_{n+1}$:
$$n\cdot l_{n+1}^2-l_{n+1}\cdot 2\sum^{n}_{i=2}l_i+\sum^{n}_{i=2}l_i^2-2\sum^{n}_{\substack{ i=2\\ i<j}}l_i\cdot l_j$$
\indent Po\v sto je u ovoj kvadratnoj koeficijent uz $l_{n+1}^2$ pozitivan, grafik ove funkcije je konveksna parabola, te joj je na segmentu $[0,l]$ maksimalna vrednost na jednom od krajeva segmenta, tj. maksimizuje se kada je $l_{n+1}$ ili 0 ili $l$.
\\
\\
\indent Analogno, svaki od $l_2, l_3,..., l_n$ tako\dj e treba biti ili 0 ili $l$. Sada treba maksimizovati ovaj izraz, pri uslovu:
$$l_2, l_3,..., l_{n+1}\in \{0, l\}.$$

\indent Posmatrana suma mo\v ze da se prezapi\v se kao:
$$\sum^{n+1}_{i=2}{l_i}^2+\sum^{n+1}_{\substack{i=2\\ i<j}}(l_i-l_j)^2=n+1$$
\indent Neka od ovih du\v zina $a\in \mathbb{N}$ ima vrednost $l$, a $n-a$ ima vrednost 0. Ubacivanjem toga u prethodno, dobija se:
$$a\cdot l^2+(n-a)a\cdot l^2=n+1$$
$$\Rightarrow l^2\cdot a(n+1-a)=n+1$$
\indent Po\v sto se $l$ minimizuje kada se $(n+1-a)a$ maksimizuje, a $(n+1-a)$ i $a$ imaju konstantan zbir, prema nejednakosti izme\dj u aritmeti\v cke i geometrijske sredine, to se de\v sava kada je $n+1-a=a$. Ovo se mo\v ze dosti\' ci kada je $n$ neparno, a kada je parno, izraz se maksimizuje kada se $a$ i $n+1-a$ razlikuju za jedan. Stoga, $a= \left \lceil{\frac{n+1}{2}}\right \rceil$.
\\
\\
\indent Time se dobija:
$$\left \lceil \frac{n+1}{2} \right \rceil \cdot \left( n-\left \lceil \frac{n+1}{2}\right \rceil +1\right)\cdot l^2=n+1$$
$$\Rightarrow l=\sqrt{\frac{n+1}{\left\lceil \frac{n+1}{2}\right\rceil \cdot \left( n-\left\lceil \frac{n+1}{2}\right\rceil+1\right)}}$$
\\
\indent Po\v sto je stranica ovog simpleksa du\v zine $\sqrt2$, odnos stranice simpleksa i udaljenosti dva $(n-1)$-dimenziona potprostora izme\dj u  kojih se taj simpleks ne mo\v ze postaviti izme\dj u njih je:
$$\frac{\sqrt2}{l}=\sqrt{\frac{2\left\lceil \frac{n+1}{2}\right\rceil\cdot \left( n-\left\lceil\frac{n+1}{2}\right\rceil+1\right)}{n+1}}$$
\\
\indent Stoga, izme\dj u dva paralelna $(n-1)$-dimenziona potprostora na udaljenosti $1$ ne mo\v ze da se postavi $n$-simpleks stranice  $\sqrt{\frac{2\left\lceil \frac{n+1}{2}\right\rceil\cdot \left( n-\left\lceil\frac{n+1}{2}\right\rceil+1\right)}{n+1}}$.
\\
\indent Dodatno, ovaj simpleks je po konstrukciji minimalan, tj. svaki pravilan $n$-simpleks kra\' ce stranice se mo\v ze postaviti izme\dj u dva paralelna $(n-1)$-dimenziona potprostora na udaljenosti 1.\qed
\\
\end{proof}
\\
\subsection[Jedan \v covek]{Jedan \v covek}
\bigskip
\begin{teo} Put koji formiraju po trojkama nekomplanarne ivice $n$-simpleksa \v cije su ivice du\v zine $\sqrt{\frac{2\left\lceil \frac{n+1}{2}\right\rceil\cdot \left( n-\left\lceil\frac{n+1}{2}\right\rceil+1\right)}{n+1}}$ je izlazni put iz trake u n dimenzija i taj put ima du\v zinu:
$$L_{n,1}=n\cdot \sqrt{\frac{2\left\lceil \frac{n+1}{2}\right\rceil\cdot \left( n-\left\lceil\frac{n+1}{2}\right\rceil+1\right)}{n+1}}.$$\end{teo}
\smallskip

\begin{proof}
Prema tvr\dj enju Teoreme 2.1, ovaj simpleks se ne mo\v ze postaviti izme\dj u posmatranih potprostora, tj. rub \v sume se\v ce simpleks, kako god ga postavili unutar nje.
\\
\\
\indent Posmatrani put je neprekidan i njegov konveksni omota\v c je ba\v s ovaj simpleks. Zato \v sto posmatrani simpleks izlazi iz ove \v sume, tj. jedan od ova dva $(n-1)$-dimenziona potprostora ga se\v ce, on mora da se\v ce i put \v ciji je taj simpleks konveksni omota\v c. Stoga, ova putanja je izlazna i ima du\v zinu:
\\
$$L_{n,1}=n\cdot \sqrt{\frac{2\left\lceil \frac{n+1}{2}\right\rceil\cdot \left( n-\left\lceil\frac{n+1}{2}\right\rceil+1\right)}{n+1}}. $$ \qed
\end{proof}
\\
\\
\indent Stoga, va\v zi slede\' ce:
$$S_1(T_n)\leqslant L_{n, 1}=n\cdot \sqrt{\frac{2\left\lceil \frac{n+1}{2}\right\rceil\cdot \left( n-\left\lceil\frac{n+1}{2}\right\rceil+1\right)}{n+1}}.$$
\\
\subsection[$\textbf{n+1}$ \v covek]{$\textbf{n+1}$ \v covek} 
\bigskip
\begin{teo} Put koji formira $n+1$ du\v zi od centra do svih temena pravilnog $n$-simpleksa  \v cije su ivice du\v zine $\sqrt{\frac{2\left\lceil \frac{n+1}{2}\right\rceil\cdot \left( n-\left\lceil\frac{n+1}{2}\right\rceil+1\right)}{n+1}}$ je izlazni put za $n+1$ ljudi iz trake u $n$ dimenzija i taj put ima du\v zinu:

$$L_{n,n+1}=\sqrt{n\cdot\left\lceil\frac{n+1}{2}\right\rceil\cdot\left( n-\left\lceil\frac{n+1}{2}\right\rceil+1\right)}.$$\end{teo}

\begin{proof}
Sli\v cno dokazu Teoreme 2.2, po\v sto svaki od $n+1$ ljudi ide od centra posmatranog simpleksa do nekog od njegovih temena, tako da do svakog temena ide ta\v cno jedan \v covek, konveksni omota\v c skupa ovih puteva je ba\v s taj simpleks. Po\v sto barem jedan od posmatranih $(n-1)$- dimenzionih potprostora se\v ce konveksni omota\v c ovog puta, barem jedan od njih mora se\' ci i taj put. 
\\
\\
\indent Odnos izme\dj u rastojanja centra simpleksa i temena sa du\v zinom ivice je:
$$\frac{\sqrt{\left( 1-\frac{1}{n+1}\right)^2+n\cdot{\frac{1}{n+1}}^2}}{\sqrt2}=\frac{\sqrt{n^2+n}}{\sqrt2 \cdot(n+1)}$$
\indent Onda je rastojanje izme\dj u centra i temena u simpleksu stranice  $\sqrt{\frac{2\left\lceil \frac{n+1}{2}\right\rceil\cdot \left( n-\left\lceil\frac{n+1}{2}\right\rceil+1\right)}{n+1}}$ je:
\\
$$L_{n,n+1}=\frac{1}{n+1}\cdot\sqrt{n\cdot\left\lceil\frac{n+1}{2}\right\rceil\cdot\left( n-\left\lceil\frac{n+1}{2}\right\rceil+1\right)}$$
\\
\indent Po\v sto svaki od $n+1$ ljudi koji kre\' cu iz centra ovog simpleksa ima put du\v zine $S$, ukupna du\v zina puta je:
\\
$$L_{n,n+1}=\sqrt{n\cdot\left\lceil\frac{n+1}{2}\right\rceil\cdot\left( n-\left\lceil\frac{n+1}{2}\right\rceil +1\right)}$$
\qed
\end{proof}
\\
\indent Stoga, va\v zi slede\' ce:
$$S_{n+1}(T_n)\leqslant L_{n,n+1}= \sqrt{n\cdot\left\lceil\frac{n+1}{2}\right\rceil\cdot\left( n-\left\lceil\frac{n+1}{2}\right\rceil +1\right)}$$


\section{Donje ograni\v cenje u $n$ dimenzija}
\bigskip
Gomi \cite{Ghomi} je pokazao donje organi\v cenje za du\v zinu puta \v sirine 1. Konstrui\v simo put za jednog \v coveka koji koriste\' ci Gomijevo ograni\v cenje, daje donje ograni\v cenje za du\v zinu modifikovanog puta.

\begin{teo}
Neka je du\v zina najkra\' ceg izlaznog puta jedini\v cne \v sirine za $k\geqslant 2$ ljudi iz trake u $n\geqslant 2$ dimenzija $S_k$. Tada va\v zi:
$$S_k\geqslant\frac{k}{2k-2}\cdot \sqrt{2.2782^2+9(n-2)}$$
\end{teo}

\begin{proof}
Neka su $\gamma_1, \gamma_2,..., \gamma_k$ putevi ovih $k$ ljudi, za svako $1\leqslant i\leqslant k$ gde va\v zi:
$$\gamma_i: [0,1]\to \mathbb{R}^n$$
\indent Cilj je da napravimo funkciju koja \' ce na neki na\v cin imitirati put za vi\v se ljudi i to radimo tako \v sto za dati put za $k$ ljudi, funkcija ,,ide" kao da kre\' ce iz krajnje ta\v cke za prvog \v coveka, zatim ,,ide" do po\v cetne ta\v cke. Zatim ,,ide" do krajnje ta\v cke drugog, pa nazad do po\v cetne, analogno i za sve do $(k-1)$-vog \v coveka.  Nakon ovoga se funkcija opet ,,nalazi" u po\v cetnoj ta\v cki i iz nje ,,ide" do krajnje ta\v cke puta za $k$-tog \v coveka.
Neka su $l_1, l_2,...,l_k$ redom du\v zine $\gamma_1, \gamma_2,..., \gamma_k$, pri uslovu da su $l_1, l_k\leqslant \min(l_2,l_3,...l_{k-1}).$

 Posmatrajmo funkciju $\gamma$ definisanu sa:
$$\gamma: [0, 2k-2]\to \mathbb{R}^n$$

\[ \gamma(x) = \begin{cases}  
          \gamma_1(1-x), & 0\leqslant x\leqslant 1; \\
          \gamma_i(x-(2i-3)), & 2i-3< x \leqslant2i-2, i\in\{2,3,...,k-1\};\\
	 \gamma_i (1-(x-(2i-2))),& 2i-2< x\leqslant 2i-1, i \in\{2,3,...,k-1\};\\
          \gamma_k(x-(2k-3)), & 2k-3<x \leqslant 2k-2.
       \end{cases}
 \]
\\
\\
\indent Po definiciji puta za $k$ ljudi, put svakog \v coveka je neprekidna funkcija i va\v zi $\gamma_1(0)=\gamma_2(0)=...=\gamma_k(0)$, te za svako $i\in \{2,3,...,k\}$ va\v zi:
$$\lim_{x\to (2i-3)^-}\gamma(x)=\gamma(2i-3)$$
$$\lim_{x\to (2i-2)^-}\gamma(x)=\gamma(2i-2)$$
\\
\indent Stoga,  $\gamma$ je neprekidna funkcija, te mo\v zemo na nju primeniti Teoremu 2.1. Primetimo da je du\v zina $\gamma$ zapravo $l_1+2\cdot(l_2+l_3+...+l_{k-1})+l_k$, te va\v zi:
$$l_1+2\cdot(l_2+l_3+...+l_{k-1})+l_k\geqslant\sqrt{2.2782^2+9(n-2)}$$
\indent Po\v sto je du\v zina posmatranog puta za $k$ ljudi $l_1+l_2+...+l_k$ i po\v sto su $l_1$ i $l_k$ najkra\' ce od ovih du\v zina, tada va\v zi:
$$l_1+l_2+...+l_k\geqslant \frac{k}{2k-2}\cdot (l_1+2\cdot(l_2+l_3+...l_{k-1})+l_k)$$
$$\Longrightarrow S_k\geqslant \frac{k}{2k-2}\cdot \sqrt{2.2782^2+9(n-2)}$$\qed
\end{proof}


\section[$k$-traka u $n$ dimenzija]{$k$-traka u $n$ dimenzija}
\bigskip
\indent Traka u $n$ dimenzija predstavlja upo\v stenje klasi\v cnog primera \v sume u $\mathbb{R}^2$ ograni\v cene sa dve paralelne prave na udaljenosti 1. Ovo je uop\v stenje koje nastaje interpretacijom ove \v sume kao prostora ograni\v cenog hiperravnima u nekom prostoru i u slu\v caju $\mathbb{R}^2$ je to jedina takva \v suma koju ima smisla posmatrati. U prostorima vi\v sih dimenzija od $\mathbb{R}^2$ to nije slu\v caj.
\\
\begin{df} Defini\v simo \textbf{k-traku u n dimenzija} za svako $n\in \mathbb{N}$ i za svako $1\leqslant k\leqslant n-1$  kao \v sumu odre\dj enu svim ta\v ckama iz $\mathbb{R}^n$ na udaljenosti manjoj ili jednakoj $\frac{1}{2}$ od nekog $k$-dimenzionog potprostora $\mathbb{R}^n$. 
\end{df}
\begin{teo} Najkra\' ca izlazna putanja iz $k$-trake u $\mathbb{R}^n$ je ista kao najkra\' ca izlazna putanja iz trake u $\mathbb{R}^{k+1}$.\end{teo}

\begin{proof}
Posmatrajmo presek nekog potprostora dimenzije $k+1$ i $k$-trake u $n$ dimenzija. Kada bilo koji objekat projektujemo na neki drugi, njihov presek mora pripadati toj projekciji. Stoga, kada bismo projektovali $k$-traku u $n$ dimenzija na posmatrani $k+1$ potprostor, ovaj presek mora pripadati toj projekciji.
\\
\\
\indent Neka je $S$ jedini\v cna sfera u $\mathbb{R}^{n-k}$ i neka je $P$ potprostor $\mathbb{R}^k$ takvi da va\v zi:

$$Lin(S) \cap Lin(P)=\emptyset,$$
gde za za skup vektora $A$ i vektorski prostor $B$, sa $Lin(A)$ ozna\v cavamo skup svih linearnih kombinacija vektora iz $A$, a sa $Lin(B)$ linearnu kombinaciju vektora baze.
\\
Tada je $k$-traka u $n$ dimenzija zapravo $Lin(S\cup P)$. 
\\
\\
\indent Po\v sto je $S$ jedini\v cne \v sirine, a $P$ ima samo $k$ dimenzija, tj. projekcija $P$ na $\mathbb{R}^{k+1}$ po $(k+1)$-voj dimenziji je \v sirine 0, linearna kombinacija njihove unije je jedini\v cne \v sirine. Stoga, projekcija $k$-trake u $n$ dimenzija na proizvoljan $\mathbb{R}^{k+1}$ potprostor je \v sirine 1, te se ne mo\v ze pokriti trakom u $(k+1)$-noj dimenziji. Zbog toga je svaki izlazni put iz trake u $(k+1)$-noj dimenziji tako\dj e i izlazni put iz $k$-trake u $n$ dimenzija.
\\
\\
\indent S druge strane, ako za proizvoljan vektor $v\in S$ defini\v semo skup:
$$X_v=\left \{ \lambda v+p \phantom{a} \middle | \phantom{a}  p\in P, |\lambda v|\leqslant \frac{1}{2}\right \},$$
tj. $X+v$ je skup dobijen tako \v sto sve vektore iz $P$ saberemo sa svim vektorima iz $S$ koji imaju isti pravac kao vektor $v$. Ovaj skup predstavlja traku u $(k+1)$-noj dimenziji, stoga va\v zi i da je izlazni put iz $k$-trake u $n$ dimenzija tako\dj  e i izlazni put iz trake u $(k+1)$-noj dimenziji. $\square$
\end{proof}
\\

\section[Traka u $\mathbb{R}^2$ sa proizvoljnim brojem ljudi]{Traka u $\mathbb{R}^2$ sa proizvoljnim brojem ljudi}
\bigskip
\begin{lem} Za \v sumu $F$ u $\mathbb{R}^n$ va\v zi: $$S_1(F)=S_2(F).$$\end{lem}

\begin{proof}
Za svaki izlazni put za jednog \v coveka postoji ekvivalentan izlazni put za dvojicu jer ako dva \v coveka krenu iz bilo koje ta\v cke tog puta i u suprotnim smerovima ga prate, obrazova\' ce isti put. Sli\v cno, za svaki izlazni put za dva \v coveka postoji ekvivalentan put za jednog \v coveka tako \v sto jedan \v covek ide putem koji bi formirao da krene iz krajnje pozicije na putu jednog od dva \v coveka i prati taj put do krajnje pozicije drugog.\qed
\end{proof}
\begin{df} Geometrijska medijana skupa ta\v caka $S=\{ X_1, X_2, ..., X_m\}$, gde su $X_i\in \mathbb{R}^n$ je ta\v cka $M\in \mathbb{R}^n$ za koju va\v zi:
$$\underset{M\in \mathbb{R}^n}{\arg\min} \sum^m_{i=1}{d(X_i, M)}, $$
gde je za $A,B\in \mathbb{R}^n$ $d(A, B)$ euklidska razdaljina izme\dj u ta\v caka $A$ i $B.$\end{df}
\begin{df} Tori\v celijeva ta\v cka je geometrijska medijana trougla. \end{df}
\indent Za Tori\v celijevu ta\v cku su poznata svojstva:
\begin{enumerate}
\item Ako je u $\bigtriangleup ABC$ jedan od uglova ve\' ci od $120^\circ$, Tori\v celijeva ta\v cka $\bigtriangleup ABC$ je teme tog ugla.
\item Ako su u $\bigtriangleup ABC$ svi uglovi ne ve\' ci od $120^\circ$,Tori\v celijeva ta\v cka $\bigtriangleup ABC$ je ta\v cka $X$ takva da va\v zi:
$$\angle AXB=\angle BXC= \angle CXA=120^\circ$$
\end{enumerate}


\begin{teo} Najkra\' ci izlazni put iz $T_2$ \v ciji je konveksni omota\v c trougao je du\v zine $2$. Ta\v cnije, to je put za troje ljudi koji kre\' cu iz centra jednakostrani\v cnog trougla i svaki \v covek ide ka jednom od temena.
\end{teo}

\begin{proof}
Neka je konveksni omota\v c posmatranog puta $\bigtriangleup ABC$ i neka je po\v cetna ta\v cka puta ta\v cka $X$ (unutar $\bigtriangleup ABC$). Da bi ovo bio konveksni omota\v c posmatranog puta, neki od ljudi koji kre\' cu iz $X$ svakako moraju da u nekom trenutku pro\dj u i kroz ta\v cku $A$, $B$ i $C$. Kada bi iz ta\v cke $X$ kretalo vi\v se od 3 osobe, svakako neka tri \v coveka (ili manje) od njih moraju da pro\dj u kroz ova tri temena trougla, a ostali nikako ne doprinose obliku konveksnog omota\v ca, te su bespotrebni, tj. kada svi ljudi osim te trojice koji prolaze kroz temena ne bi krenuli, put bi bio kra\' ci. Stoga, optimalno je da krene najvi\v se tri osobe. 
\\
\\
\indent Ako za $\bigtriangleup ABC$ va\v zi da su mu svi uglovi ne ve\' ci od $120^\circ$, Tori\v celijeva ta\v cka je unutar trougla i ,,gleda" sve stranice trougla pod uglom od $120^\circ$. Put koji obrazuje troje ljudi koji kre\' cu iz Tori\v celijeve ta\v cke i svaki ide do jednog temena je kra\' ci od bilo kog puta za jednog \v coveka, jer bi taj put morao da pro\dj e kroz sva temena, tj. najkra\' ci put za jednog \v coveka bi bio sa\v cinjen od dve najkra\' ce stranice, \v sto je du\v ze od puta iz Tori\v celijeve ta\v cke (po definiciji geometrijske medijane).
\\
\\
\indent Stoga, neka je $X$ Tori\v celijeva ta\v cka $\bigtriangleup ABC$. Neka su $h_a, h_b, h_c$ redom visine iz temena $A, B, C$ i neka je $x=|AX|, y=|BX|, z=|CX|$. Da bi put sa ovim konveksnim omota\v cem bio izlazan iz $T_2$, mora va\v ziti $\min{\{h_a,h_b,h_c\}}\geqslant 1$, s tim da se tra\v zi minimalna suma $x+y+z$ pri kojoj to va\v zi.
\\
\\
\indent Problem minimizacije $x+y+z$ pri uslovu $\min{\{h_a,h_b,h_c\}}\geqslant 1$ je ekvivalentan problemu maksimizacije $\min{\{h_a,h_b,h_c\}}$ ako fiksiramo $x+y+z=c$, za bilo koju pozitivnu konstantu $c$.
\\
\\
\indent Bez umanjenja op\v steg, neka je $h_a\leqslant h_b\leqslant h_c$, tj. $x\leqslant y\leqslant z$. Va\v zi slede\' ce:
\begin{center}
$$P(\bigtriangleup ABC)=\frac{h_a\cdot BC}{2}=P(\bigtriangleup ABX)+P(\bigtriangleup BCX)+P(\bigtriangleup CAX)=$$
$$=\frac{xy\cdot \sin{(\angle AXB)}}{2}+\frac{yz\cdot \sin{(\angle BXC)}}{2}+\frac{zx\cdot \sin(\angle CXA)}{2}=\frac{\sqrt3}{4}\cdot(xy+yz+zx)$$
$$\Longrightarrow h_a=\frac{\sqrt3}{2}\cdot \frac{xy+yz+zx}{BC}.$$
\end{center}
\end{proof} 
\smallskip

\indent Kosinusna teorema primenjena na $\bigtriangleup BCX$ daje:
\\
$$BC=\sqrt{y^2+z^2-2\cos{(\angle BXC)}\cdot yz}=\sqrt{y^2+z^2+yz}.$$
\\
\indent Ubacivanjem izraza za $BC$ u izraz za $h_a$ dobijamo izraz koji treba maksimizovati  pri uslovu $x\leqslant y\leqslant z$:
$$h_a=\frac{\sqrt3}{2}\cdot \frac{xy+yz+zx}{\sqrt{y^2+z^2+yz}}\leqslant \frac{\sqrt3}{2}\cdot \frac{xy+yz+zx}{\sqrt{xy+yz+zx}}=\frac{\sqrt3}{2}\cdot \sqrt{xy+yz+zx}.$$
\\
\indent Po\v sto je $x+y+z=c$ i ovi brojevi predstavljaju du\v zine, tj. nenegativni su, prema nejednakosti izme\dj u aritmeti\v cke i geometrijske sredine va\v zi:
$$xy+yz+zx\leqslant x^2+y^2+z^2$$
$$\Longrightarrow xy+yz+zx\leqslant \frac{1}{3}\cdot (x+y+z)^2=\frac{1}{3}\cdot c^2$$
$$\Longrightarrow \sqrt{xy+yz+zx}\leqslant \frac{1}{\sqrt3}\cdot c.$$
\\
\indent Maksimum se dosti\v ze za $x=y=z$ i tada je:
$$1=\frac{\sqrt3}{2}\sqrt{xy+yz+zx}=\frac{1}{2}\cdot c,$$
tj. $c=2$. Ovo je slu\v caj gde je $\bigtriangleup ABC$ jednakostrani\v can i $X$ je njegov centar.
\\
\\
\indent Ako je neki od uglova $\bigtriangleup ABC$ ve\' ci od $120^\circ$, Tori\v celijeva ta\v cka je ono teme u kom se spajaju dve najkra\' ce stranice. Zbog pretpostavle $h_a\leqslant h_b\leqslant h_c$ va\v zi $\angle BAC>120^\circ$.
\\
\\
\indent Ponovo je uslov da ovaj put bude izlazan iz $T_2$ da su sve visine ovog trougla du\v zine barem 1, tj. da je $h_a\geqslant 1$ i va\v zi:
$$P(\bigtriangleup ABC)=\frac{1}{2} h_a \cdot BC=\frac{1}{2} \cdot AB\cdot CA \cdot \sin(\angle BAC)$$
$$\Longrightarrow h_a=\frac{ AB\cdot CA \cdot \sin(\angle BAC)}{BC}=\sin{(\angle CBA)}\cdot AB\geqslant 1$$
\\
\indent Po\v sto je $\angle BAC>120^\circ$, tada je $\angle CBA<30^\circ$, tj. $\sin(\angle CBA)<\frac{1}{2}$, mora va\v ziti $AB>2$, te mora va\v ziti i $AB+CA>2$, te ovaj slu\v caj ne mo\v ze biti bolji od prethodnog, \v cime je dokaz kompletan.\qed
\\
\section[Zaklju\v cak]{Zaklju\v cak}
\bigskip
U ovom radu smo se primarno bavili uvo\dj enjem i ispitivanjem modifikacije Belmanovog problema izlaska iz \v sume, sa akcentom na krive u $\mathbb{R}^n$ jedini\v cne \v sirine, za \v ciju smo minimalnu du\v zinu dali dva gornja ograni\v cenja (u formi primera puteva) - za jednog \v coveka i za $(n+1)$-nog \v coveka, kao i donje ograni\v cenje u zavisnosti od broja ljudi.
\\
\\
\indent Dalji rad bi mogao da se bavi tra\v zenjem grani\v cne vrednosti du\v zine najkra\' ceg puta \v sirine 1 u $\mathbb{R}^n$ sa $k$ ljudi, kada $k\to \infty$. U odeljku 1.3 je argumentovano za\v sto ta grani\v cna vrednost postoji za svaku \v sumu. Hipoteza 2.2 implicira da smatramo da \' ce ta grani\v cna vrednost biti dostignuta ba\v s za put za $(n+1)$-nog \v coveka konstruisan u odeljku 2.3. Shodno tome, o\v cekujemo da se donje ograni\v cenje za ovu vrednost mo\v ze dosta poo\v striti i u slu\v caju jednog \v coveka, i vi\v se ljudi.
\\
\\
\indent Specijalno, o\v cekujemo da \' ce u ravni najkra\' ci put za proizvoljan broj ljudi biti tri \v coveka koja idu pravo du\v zima du\v zine $\frac{2}{3}$, pod uglom od $120^\circ$. Ideja teoreme 5.4, koja tvrdi da je ovaj put najkra\' ci od svih puteva u ravni \v sirine 1 \v ciji je konveksni omota\v c trougao, jeste da se mo\v zda za svaki put \v ciji je konveksan omota\v c neka figura $F$ \v sirine 1, mo\v ze na\' ci trougao \v sirine 1 u kom postoji kra\' ci put.
\\
\\
\indent Znamo da krive konstruisane u odeljku 2.2 nisu optimalne za jednog \v coveka. Naime, lako se mogu skratiti, npr. \v covek mo\v ze i\' ci ivicom ovog simpleksa dok ne do\dj e do du\v zine 1 i zatim pre\' ci na slede\' cu ivicu putem normalnim na nju. Zalgalerove krive u ravni i u prostoru predstavljaju neku vrstu zakrivljenja puteva po simpleksu opisanih u odeljku 2.2, te smatramo da se i u $\mathbb{R}^n$ mo\v ze na sli\v can na\v cin dobiti kriva kra\' ca od ove, \v cija je \v sirina 1.
\bigskip
\bigskip

\begin{thebibliography}{99}


\bibitem{Bellman}
	R. Bellman,
	\textit{Minimization Problem},
	Bulletin of the American Mathematical Society, 62: 270  (1956)
\bibitem{Besikovic}
	A. S. Besicovitch,
	\textit{On arcs that cannot be covered by an open equiteral triangle of side 1},
	Mathematical Gazette 49: 286-288 (1964)

\bibitem{Finch}
	S. R. Finch, J. E. Wetzel,
	\textit{Lost in a forest},
	The American Mathematical Monthly, 111(8) : 645-654 (2004)
\bibitem{Zalgaller3d}
	S. R. Finch,
	\textit{A translation of Zalgaller's "The shortest space curve of unit width" (1994)}
	arXiv preprint(2019), arXiv:1910.02729
\bibitem{Debele}
	J. Gerriets, G. Poole,
	\textit{Convex Regions Which Cover Arcs of Constant Length},
	The American Mathematical Monthly, 81: 36-41 (1974)
\bibitem{Ghomi}
 	M. Ghomi,
	\textit{The length, width, and inradius of space curves},
	arXiv preprint(2018), arXiv: 1605.01144v3



\bibitem{Gibbs}
	P. Gibbs, 
	\textit{Bellman’s Escape Problem for Convex Polygons},
	viXra:1606.0050 (2016)

\bibitem{Gibbs2}
	P. Gibbs,
	\textit{Lost in a Isosceles Triangle},
	viXra:1606.0015 (2016)

\bibitem{Wetzel}
	Y. Movshovich, J. E. Wetzel,
	\textit{Drapeable unit arcs fit in the unit $30^\circ$ sector},
	 Adv. Geom. 17: 497–506 (2017)

\bibitem{SVE}
    J. W. Ward,
    \textit{Exploring the Bellman Forest Problem},
    Spring (2008)

\bibitem{Zalgaller}
	V. A. Zalgaller,
	\textit{How to get out of the woods? On a problem of Bellman},
	Matematicheskoe Prosveshchenie 6: 191-195 (1961)


\end{thebibliography}


\end{document}













